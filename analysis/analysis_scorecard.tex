% Options for packages loaded elsewhere
\PassOptionsToPackage{unicode}{hyperref}
\PassOptionsToPackage{hyphens}{url}
%
\documentclass[
]{article}
\usepackage{amsmath,amssymb}
\usepackage{lmodern}
\usepackage{iftex}
\ifPDFTeX
  \usepackage[T1]{fontenc}
  \usepackage[utf8]{inputenc}
  \usepackage{textcomp} % provide euro and other symbols
\else % if luatex or xetex
  \usepackage{unicode-math}
  \defaultfontfeatures{Scale=MatchLowercase}
  \defaultfontfeatures[\rmfamily]{Ligatures=TeX,Scale=1}
\fi
% Use upquote if available, for straight quotes in verbatim environments
\IfFileExists{upquote.sty}{\usepackage{upquote}}{}
\IfFileExists{microtype.sty}{% use microtype if available
  \usepackage[]{microtype}
  \UseMicrotypeSet[protrusion]{basicmath} % disable protrusion for tt fonts
}{}
\makeatletter
\@ifundefined{KOMAClassName}{% if non-KOMA class
  \IfFileExists{parskip.sty}{%
    \usepackage{parskip}
  }{% else
    \setlength{\parindent}{0pt}
    \setlength{\parskip}{6pt plus 2pt minus 1pt}}
}{% if KOMA class
  \KOMAoptions{parskip=half}}
\makeatother
\usepackage{xcolor}
\usepackage[margin=1in]{geometry}
\usepackage{color}
\usepackage{fancyvrb}
\newcommand{\VerbBar}{|}
\newcommand{\VERB}{\Verb[commandchars=\\\{\}]}
\DefineVerbatimEnvironment{Highlighting}{Verbatim}{commandchars=\\\{\}}
% Add ',fontsize=\small' for more characters per line
\usepackage{framed}
\definecolor{shadecolor}{RGB}{248,248,248}
\newenvironment{Shaded}{\begin{snugshade}}{\end{snugshade}}
\newcommand{\AlertTok}[1]{\textcolor[rgb]{0.94,0.16,0.16}{#1}}
\newcommand{\AnnotationTok}[1]{\textcolor[rgb]{0.56,0.35,0.01}{\textbf{\textit{#1}}}}
\newcommand{\AttributeTok}[1]{\textcolor[rgb]{0.77,0.63,0.00}{#1}}
\newcommand{\BaseNTok}[1]{\textcolor[rgb]{0.00,0.00,0.81}{#1}}
\newcommand{\BuiltInTok}[1]{#1}
\newcommand{\CharTok}[1]{\textcolor[rgb]{0.31,0.60,0.02}{#1}}
\newcommand{\CommentTok}[1]{\textcolor[rgb]{0.56,0.35,0.01}{\textit{#1}}}
\newcommand{\CommentVarTok}[1]{\textcolor[rgb]{0.56,0.35,0.01}{\textbf{\textit{#1}}}}
\newcommand{\ConstantTok}[1]{\textcolor[rgb]{0.00,0.00,0.00}{#1}}
\newcommand{\ControlFlowTok}[1]{\textcolor[rgb]{0.13,0.29,0.53}{\textbf{#1}}}
\newcommand{\DataTypeTok}[1]{\textcolor[rgb]{0.13,0.29,0.53}{#1}}
\newcommand{\DecValTok}[1]{\textcolor[rgb]{0.00,0.00,0.81}{#1}}
\newcommand{\DocumentationTok}[1]{\textcolor[rgb]{0.56,0.35,0.01}{\textbf{\textit{#1}}}}
\newcommand{\ErrorTok}[1]{\textcolor[rgb]{0.64,0.00,0.00}{\textbf{#1}}}
\newcommand{\ExtensionTok}[1]{#1}
\newcommand{\FloatTok}[1]{\textcolor[rgb]{0.00,0.00,0.81}{#1}}
\newcommand{\FunctionTok}[1]{\textcolor[rgb]{0.00,0.00,0.00}{#1}}
\newcommand{\ImportTok}[1]{#1}
\newcommand{\InformationTok}[1]{\textcolor[rgb]{0.56,0.35,0.01}{\textbf{\textit{#1}}}}
\newcommand{\KeywordTok}[1]{\textcolor[rgb]{0.13,0.29,0.53}{\textbf{#1}}}
\newcommand{\NormalTok}[1]{#1}
\newcommand{\OperatorTok}[1]{\textcolor[rgb]{0.81,0.36,0.00}{\textbf{#1}}}
\newcommand{\OtherTok}[1]{\textcolor[rgb]{0.56,0.35,0.01}{#1}}
\newcommand{\PreprocessorTok}[1]{\textcolor[rgb]{0.56,0.35,0.01}{\textit{#1}}}
\newcommand{\RegionMarkerTok}[1]{#1}
\newcommand{\SpecialCharTok}[1]{\textcolor[rgb]{0.00,0.00,0.00}{#1}}
\newcommand{\SpecialStringTok}[1]{\textcolor[rgb]{0.31,0.60,0.02}{#1}}
\newcommand{\StringTok}[1]{\textcolor[rgb]{0.31,0.60,0.02}{#1}}
\newcommand{\VariableTok}[1]{\textcolor[rgb]{0.00,0.00,0.00}{#1}}
\newcommand{\VerbatimStringTok}[1]{\textcolor[rgb]{0.31,0.60,0.02}{#1}}
\newcommand{\WarningTok}[1]{\textcolor[rgb]{0.56,0.35,0.01}{\textbf{\textit{#1}}}}
\usepackage{graphicx}
\makeatletter
\def\maxwidth{\ifdim\Gin@nat@width>\linewidth\linewidth\else\Gin@nat@width\fi}
\def\maxheight{\ifdim\Gin@nat@height>\textheight\textheight\else\Gin@nat@height\fi}
\makeatother
% Scale images if necessary, so that they will not overflow the page
% margins by default, and it is still possible to overwrite the defaults
% using explicit options in \includegraphics[width, height, ...]{}
\setkeys{Gin}{width=\maxwidth,height=\maxheight,keepaspectratio}
% Set default figure placement to htbp
\makeatletter
\def\fps@figure{htbp}
\makeatother
\setlength{\emergencystretch}{3em} % prevent overfull lines
\providecommand{\tightlist}{%
  \setlength{\itemsep}{0pt}\setlength{\parskip}{0pt}}
\setcounter{secnumdepth}{-\maxdimen} % remove section numbering
\usepackage{booktabs}
\usepackage{longtable}
\usepackage{array}
\usepackage{multirow}
\usepackage{wrapfig}
\usepackage{float}
\usepackage{colortbl}
\usepackage{pdflscape}
\usepackage{tabu}
\usepackage{threeparttable}
\usepackage{threeparttablex}
\usepackage[normalem]{ulem}
\usepackage{makecell}
\usepackage{xcolor}
\ifLuaTeX
  \usepackage{selnolig}  % disable illegal ligatures
\fi
\IfFileExists{bookmark.sty}{\usepackage{bookmark}}{\usepackage{hyperref}}
\IfFileExists{xurl.sty}{\usepackage{xurl}}{} % add URL line breaks if available
\urlstyle{same} % disable monospaced font for URLs
\hypersetup{
  pdftitle={analysis},
  pdfauthor={Brad Stirling},
  hidelinks,
  pdfcreator={LaTeX via pandoc}}

\title{analysis}
\author{Brad Stirling}
\date{2022-07-30}

\begin{document}
\maketitle

\begin{Shaded}
\begin{Highlighting}[]
\FunctionTok{library}\NormalTok{(multcomp)}
\end{Highlighting}
\end{Shaded}

\begin{verbatim}
## Loading required package: mvtnorm
\end{verbatim}

\begin{verbatim}
## Loading required package: survival
\end{verbatim}

\begin{verbatim}
## Loading required package: TH.data
\end{verbatim}

\begin{verbatim}
## Loading required package: MASS
\end{verbatim}

\begin{verbatim}
## 
## Attaching package: 'TH.data'
\end{verbatim}

\begin{verbatim}
## The following object is masked from 'package:MASS':
## 
##     geyser
\end{verbatim}

\begin{Shaded}
\begin{Highlighting}[]
\FunctionTok{library}\NormalTok{(fixest)}
\FunctionTok{library}\NormalTok{(tidyverse)}
\end{Highlighting}
\end{Shaded}

\begin{verbatim}
## -- Attaching packages --------------------------------------- tidyverse 1.3.1 --
\end{verbatim}

\begin{verbatim}
## v ggplot2 3.3.6     v purrr   0.3.4
## v tibble  3.1.7     v dplyr   1.0.9
## v tidyr   1.2.0     v stringr 1.4.0
## v readr   2.1.2     v forcats 0.5.1
\end{verbatim}

\begin{verbatim}
## -- Conflicts ------------------------------------------ tidyverse_conflicts() --
## x dplyr::filter() masks stats::filter()
## x dplyr::lag()    masks stats::lag()
## x dplyr::select() masks MASS::select()
\end{verbatim}

\begin{Shaded}
\begin{Highlighting}[]
\FunctionTok{library}\NormalTok{(vtable)}
\end{Highlighting}
\end{Shaded}

\begin{verbatim}
## Loading required package: kableExtra
\end{verbatim}

\begin{verbatim}
## Warning in !is.null(rmarkdown::metadata$output) && rmarkdown::metadata$output
## %in% : 'length(x) = 2 > 1' in coercion to 'logical(1)'
\end{verbatim}

\begin{verbatim}
## 
## Attaching package: 'kableExtra'
\end{verbatim}

\begin{verbatim}
## The following object is masked from 'package:dplyr':
## 
##     group_rows
\end{verbatim}

\begin{Shaded}
\begin{Highlighting}[]
\FunctionTok{load}\NormalTok{(}\StringTok{"processed\_data/processed\_scorecard.RData"}\NormalTok{)}
\end{Highlighting}
\end{Shaded}

\hypertarget{introduction-and-data-cleaning-overview}{%
\subsection{Introduction and Data Cleaning
Overview}\label{introduction-and-data-cleaning-overview}}

This analysis explores the impact of the publication of the college
scorecard on consumer interest in bachelor's degree-granting educational
institutions. Specifically, it examines the relationship with 10-year
post graduate earnings. The goal was to see if the scorecard shifted
interest from schools with low post-graduate earnings to schools with
high ones. In addition to post graduate earnings, the analysis explored
controls across intuition types, average student age, percentage of STEM
programs, SAT averages, and median debt

Consumer interest was measured by looking at the average shift in
standardized Google search rankings following the publication of the
scorecard in September 2015. As the Google trends data had 7 months of
post-scorecard observations, the table was filtered to 14 months for the
analysis. To account for the seasonality of the college application
deadlines, the 7-month pre-scorecard covered the time frame of September
2014 to March 2015. An additional binary variable was created to
categorize pre and post scorecard implementation groups.

The search rankings were then standardized for each school name and
keyword. A histogram of the resulting z-scores showed a fairly
normal-looking distribution. Next, means were created for both pre and
post scorecard implementation for each school. The data set was then
grouped by school and scorecard implementation category and z-score
means were calculated. The z-score means were then un-pivoted in
separate columns for pre and post z-score means, and the difference
between the post and pre z-score rankings were added. This resulted in a
trends table with a single row for each school with their corresponding
scorecard-period z-score means and the post-scorecard ranking shift. The
post-scorecard ranking shift was used as the outcome variable of the
regression model.

Next, the scorecard data set was filtered to only predominantly
bachelor's degree-granting schools. Variables that were identified as
potential controls, such as institution type, program emphasis, test
scores, age, and debt were converted into either decimals or factors for
regression testing. Given the potential for Following the initial clean
up, the school id link table was filtered to remove duplicate schools
with duplicate names, and inner-joined to both the scorecard and the
summarized school-level trend data.

To determine the categorization of high-earning vs low-earning schools,
the quantile function was used on the median-earnings 10 years
post-graduation column. Schools that had values less than the 25\%
quantile (\$35500) were labeled as low, while schools greater than the
75\% quantile (\$48100). The data set was then filtered to remove the
other values, and the high-earning vs low-earning variable was
re-categorized as binary, with a value of 1 indicating a high-earning
school. The high-earning school binary category was used as the
treatment variable of the regression model.

\hypertarget{data-exploration}{%
\subsection{Data Exploration}\label{data-exploration}}

Following the initial data cleaning and creation of the treatment and
outcome variables, control variables identified from the data dictionary
were further explored as potential sources of endogeneity.

\includegraphics{analysis_scorecard_files/figure-latex/unnamed-chunk-3-1.pdf}

\begin{Shaded}
\begin{Highlighting}[]
\NormalTok{scorecard\_f }\SpecialCharTok{\%\textgreater{}\%} \FunctionTok{filter}\NormalTok{(med\_earnings\_10yrs }\SpecialCharTok{\textgreater{}} \DecValTok{100000}\NormalTok{) }\SpecialCharTok{\%\textgreater{}\%} \FunctionTok{arrange}\NormalTok{(}\FunctionTok{desc}\NormalTok{(med\_earnings\_10yrs))}
\end{Highlighting}
\end{Shaded}

\begin{verbatim}
## # A tibble: 6 x 128
##   UNITID  OPEID opeid6 INSTNM          CITY  STABBR INSTURL NPCURL  HCM2 PREDDEG
##    <dbl>  <dbl>  <dbl> <chr>           <chr> <chr>  <chr>   <chr>  <dbl>   <dbl>
## 1 145558 165900   1659 Rosalind Frank~ Nort~ IL     www.ro~ NULL       0       3
## 2 196255 283900   2839 SUNY Downstate~ Broo~ NY     www.do~ NULL       0       3
## 3 166656 216500   2165 MCPHS Universi~ Bost~ MA     www.mc~ www.m~     0       3
## 4 196307 284000   2840 Upstate Medica~ Syra~ NY     www.up~ www.s~     0       3
## 5 122296 701200   7012 Samuel Merritt~ Oakl~ CA     www.sa~ NULL       0       3
## 6 188526 288500   2885 Albany College~ Alba~ NY     www.ac~ www.a~     0       3
## # ... with 118 more variables: institution_type <fct>, LOCALE <fct>,
## #   HBCU <chr>, PBI <chr>, ANNHI <chr>, TRIBAL <chr>, AANAPII <chr>, HSI <chr>,
## #   NANTI <chr>, MENONLY <chr>, WOMENONLY <chr>, RELAFFIL <chr>, SATVR25 <chr>,
## #   SATVR75 <chr>, SATMT25 <chr>, SATMT75 <chr>, SATWR25 <chr>, SATWR75 <chr>,
## #   SATVRMID <chr>, SATMTMID <chr>, SATWRMID <chr>, ACTCM25 <chr>,
## #   ACTCM75 <chr>, ACTEN25 <chr>, ACTEN75 <chr>, ACTMT25 <chr>, ACTMT75 <chr>,
## #   ACTWR25 <chr>, ACTWR75 <chr>, ACTCMMID <chr>, ACTENMID <chr>, ...
\end{verbatim}

After reviewing the histogram of colleges with earnings greater than
\$100K, it was observed that the majority of their degrees were awarded
in STEM-related categories. As STEM fields were already being
popularized for careers in high-paying professions, there was a
likelihood that the release of the scorecard would not further influence
consumer search behavior, and thus could bias the high-earning treatment
variable. To address this, the STEM degree percentage categories were
summed, and a binary variable was added, separating out schools with
STEM graduates greater than 50\%.

\begin{Shaded}
\begin{Highlighting}[]
\NormalTok{sat\_nulls }\OtherTok{\textless{}{-}} \FunctionTok{sum}\NormalTok{(}\FunctionTok{is.na}\NormalTok{(scorecard\_f}\SpecialCharTok{$}\NormalTok{SAT\_AVG))}
\FunctionTok{sprintf}\NormalTok{(}\StringTok{"Missing SAT score rows: \%s"}\NormalTok{, sat\_nulls)}
\end{Highlighting}
\end{Shaded}

\begin{verbatim}
## [1] "Missing SAT score rows: 354"
\end{verbatim}

Average SAT scores were considered as another possible source of bias,
but were dropped after reviewing the number of missing entries. It was
concluded that the STEM-majority school variable would control for this
as high SAT scores would be necessary for admission to these programs.

\begin{Shaded}
\begin{Highlighting}[]
\NormalTok{scorecard\_f }\SpecialCharTok{\%\textgreater{}\%} \FunctionTok{group\_by}\NormalTok{(institution\_type) }\SpecialCharTok{\%\textgreater{}\%} \FunctionTok{summarize}\NormalTok{(}\AttributeTok{count =} \FunctionTok{n}\NormalTok{())}
\end{Highlighting}
\end{Shaded}

\begin{verbatim}
## # A tibble: 3 x 2
##   institution_type count
##   <fct>            <int>
## 1 Public             215
## 2 PrivateNonProfit   505
## 3 PrivateForProfit   166
\end{verbatim}

Institution type (private for profit, private non-profit, and public)
was identified for inclusion as a fixed effect variable. As private
universities are typically more expensive than public universities, this
could also control for average grad school debt, cost, percentage of
Pell loans, and percentage of federal loans. The higher cost of the
university could also be correlated with the existing expectation that
the education would lead to higher post-graduation earnings. This could
bias the high-earning treatment variable as individuals who attend these
schools could be more driven to seek out higher paying jobs after
graduation, regardless of the impact of attending the school.

The final variable that was selected to potentially control for
endogeneity was the percentage of undergraduates aged 25 or older. Older
graduates could have multiple characteristics that could bias the
high-earning treatment variable. They could have more focused career
goals resulting in greater financial success after graduation. Their
additional work experience could increase could lead to higher incomes
afterward as they would already have stronger resumes.

\hypertarget{model-analysis}{%
\subsection{Model Analysis}\label{model-analysis}}

\begin{Shaded}
\begin{Highlighting}[]
\NormalTok{m1 }\OtherTok{\textless{}{-}}\NormalTok{ scorecard\_f }\SpecialCharTok{\%\textgreater{}\%} \FunctionTok{feols}\NormalTok{(ind\_z\_shift }\SpecialCharTok{\textasciitilde{}}\NormalTok{ high\_earning, }\AttributeTok{se =} \StringTok{"hetero"}\NormalTok{)}
\FunctionTok{etable}\NormalTok{(m1)}
\end{Highlighting}
\end{Shaded}

\begin{verbatim}
##                                  m1
## Dependent Var.:         ind_z_shift
##                                    
## (Intercept)     -0.2138*** (0.0184)
## high_earning    -0.1483*** (0.0295)
## _______________ ___________________
## S.E. type       Heteroskedast.-rob.
## Observations                    886
## R2                          0.02777
## Adj. R2                     0.02667
## ---
## Signif. codes: 0 '***' 0.001 '**' 0.01 '*' 0.05 '.' 0.1 ' ' 1
\end{verbatim}

An initial model was created as a baseline, regressing the standardized
post-scorecard search ranking shift on the high-earning treatment. The
model shows that low-earnings schools saw a increase in average search
rankings of -0.21 standard deviations following the introduction of the
scorecard (as the rankings were on a scale staring an 1, a decrease in
rank is equivalent to an increase). High-earning schools are linearly
associated with an additional -0.15 standard deviation increase in
search rankings post-scorecard. Both the intercept and the treatment are
statistically significant at an alpha of 0.001, meaning that if the null
hypothesis were true, we would only get results like these in 0.1\% of
the samples.

\begin{Shaded}
\begin{Highlighting}[]
\NormalTok{m2 }\OtherTok{\textless{}{-}}\NormalTok{ scorecard\_f }\SpecialCharTok{\%\textgreater{}\%} \FunctionTok{feols}\NormalTok{(ind\_z\_shift }\SpecialCharTok{\textasciitilde{}}\NormalTok{ high\_earning }\SpecialCharTok{|}\NormalTok{ institution\_type, }\AttributeTok{se =} \StringTok{"hetero"}\NormalTok{)}

\FunctionTok{etable}\NormalTok{(m1, m2)}
\end{Highlighting}
\end{Shaded}

\begin{verbatim}
##                                   m1                  m2
## Dependent Var.:          ind_z_shift         ind_z_shift
##                                                         
## (Intercept)      -0.2138*** (0.0184)                    
## high_earning     -0.1483*** (0.0295) -0.1011*** (0.0244)
## Fixed-Effects:   ------------------- -------------------
## institution_type                  No                 Yes
## ________________ ___________________ ___________________
## S.E. type        Heteroskedast.-rob. Heteroskedast.-rob.
## Observations                     886                 886
## R2                           0.02777             0.31493
## Within R2                         --             0.01831
## ---
## Signif. codes: 0 '***' 0.001 '**' 0.01 '*' 0.05 '.' 0.1 ' ' 1
\end{verbatim}

Next, institution type was added as fixed effects variable. This caused
the effect of the high-earning coefficient to drop to -0.10, meaning
controlling for the other variables in the model, high-earning schools
saw an increase of -0.10 standard deviation search rankings in addition
to low-earning schools following the release of the scorecard. Also, 2\%
of the variation in the search-ranking shift post scorecard was
explained by the within variation in the high-earning treatment, while
31\% was explained by the within variation and the institution type.

\begin{Shaded}
\begin{Highlighting}[]
\NormalTok{m3 }\OtherTok{\textless{}{-}}\NormalTok{ scorecard\_f }\SpecialCharTok{\%\textgreater{}\%} \FunctionTok{feols}\NormalTok{(ind\_z\_shift }\SpecialCharTok{\textasciitilde{}}\NormalTok{ high\_earning }\SpecialCharTok{+}\NormalTok{ stem\_pri }\SpecialCharTok{|}\NormalTok{ institution\_type, }\AttributeTok{se =} \StringTok{"hetero"}\NormalTok{)}

\FunctionTok{etable}\NormalTok{(m1, m2, m3)}
\end{Highlighting}
\end{Shaded}

\begin{verbatim}
##                                   m1                  m2                  m3
## Dependent Var.:          ind_z_shift         ind_z_shift         ind_z_shift
##                                                                             
## (Intercept)      -0.2138*** (0.0184)                                        
## high_earning     -0.1483*** (0.0295) -0.1011*** (0.0244) -0.1086*** (0.0245)
## stem_priTRUE                                                 0.0676 (0.0411)
## Fixed-Effects:   ------------------- ------------------- -------------------
## institution_type                  No                 Yes                 Yes
## ________________ ___________________ ___________________ ___________________
## S.E. type        Heteroskedast.-rob. Heteroskedast.-rob. Heteroskedast.-rob.
## Observations                     886                 886                 886
## R2                           0.02777             0.31493             0.31754
## Within R2                         --             0.01831             0.02205
## ---
## Signif. codes: 0 '***' 0.001 '**' 0.01 '*' 0.05 '.' 0.1 ' ' 1
\end{verbatim}

The STEM-graduate majority variable was then added as a binary control
while keeping institution type as a fixed effect variable. Holding
constant all other variables, STEM-focused schools saw a drop of 0.07
standard deviation rankings after the publication of the scorecard, but
was not a statistically significant result. The resulting shift in the
high-earning coefficient boosted the search trend rankings by -0.01
standard deviations, while the overall variation in the model improved
by 1\%. Due to the low impact of STEM-graduate majority variable, it was
dropped from further regressions.

\begin{Shaded}
\begin{Highlighting}[]
\NormalTok{m4 }\OtherTok{\textless{}{-}}\NormalTok{ scorecard\_f }\SpecialCharTok{\%\textgreater{}\%} \FunctionTok{feols}\NormalTok{(ind\_z\_shift }\SpecialCharTok{\textasciitilde{}}\NormalTok{ high\_earning }\SpecialCharTok{+}\NormalTok{ UG25abv }\SpecialCharTok{|}\NormalTok{ institution\_type, }\AttributeTok{se =} \StringTok{"hetero"}\NormalTok{)}
\end{Highlighting}
\end{Shaded}

\begin{verbatim}
## NOTE: 4 observations removed because of NA values (RHS: 4).
\end{verbatim}

\begin{Shaded}
\begin{Highlighting}[]
\FunctionTok{etable}\NormalTok{(m1, m2, m4)}
\end{Highlighting}
\end{Shaded}

\begin{verbatim}
##                                   m1                  m2                  m4
## Dependent Var.:          ind_z_shift         ind_z_shift         ind_z_shift
##                                                                             
## (Intercept)      -0.2138*** (0.0184)                                        
## high_earning     -0.1483*** (0.0295) -0.1011*** (0.0244) -0.0985*** (0.0235)
## UG25abv                                                  -0.5085*** (0.0614)
## Fixed-Effects:   ------------------- ------------------- -------------------
## institution_type                  No                 Yes                 Yes
## ________________ ___________________ ___________________ ___________________
## S.E. type        Heteroskedast.-rob. Heteroskedast.-rob. Heteroskedast.-rob.
## Observations                     886                 886                 882
## R2                           0.02777             0.31493             0.37027
## Within R2                         --             0.01831             0.09625
## ---
## Signif. codes: 0 '***' 0.001 '**' 0.01 '*' 0.05 '.' 0.1 ' ' 1
\end{verbatim}

Next, the percentage of students aged 25 or older was added. Adjusting
for other variables in the model, a 1 unit increase in the percentage of
the 25+ aged students resulted in a search rank increase of -0.51
standard deviations. The impact of high-earning schools was in line with
the earlier model, with interest via search trend standard deviations
-0.1 higher than low-earning schools. The overall variation explained in
the model rose to 37\% with 10\% explained by the within variation in
the high-earning and percentage of 25+ aged students.

\begin{verbatim}
## `geom_smooth()` using formula 'y ~ x'
\end{verbatim}

\begin{verbatim}
## Warning: Removed 4 rows containing non-finite values (stat_smooth).
\end{verbatim}

\begin{verbatim}
## Warning: Removed 4 rows containing missing values (geom_point).
\end{verbatim}

\includegraphics{analysis_scorecard_files/figure-latex/unnamed-chunk-11-1.pdf}

\begin{Shaded}
\begin{Highlighting}[]
\NormalTok{m5 }\OtherTok{\textless{}{-}}\NormalTok{ scorecard\_f }\SpecialCharTok{\%\textgreater{}\%} \FunctionTok{feols}\NormalTok{(ind\_z\_shift }\SpecialCharTok{\textasciitilde{}}\NormalTok{ high\_earning }\SpecialCharTok{+}\NormalTok{ UG25abv }\SpecialCharTok{+}\NormalTok{ high\_earning }\SpecialCharTok{*}\NormalTok{ UG25abv}\SpecialCharTok{|}\NormalTok{ institution\_type, }\AttributeTok{se =} \StringTok{"hetero"}\NormalTok{)}
\end{Highlighting}
\end{Shaded}

\begin{verbatim}
## NOTE: 4 observations removed because of NA values (RHS: 4).
\end{verbatim}

\begin{Shaded}
\begin{Highlighting}[]
\FunctionTok{etable}\NormalTok{(m1, m2, m4, m5)}
\end{Highlighting}
\end{Shaded}

\begin{verbatim}
##                                         m1                  m2
## Dependent Var.:                ind_z_shift         ind_z_shift
##                                                               
## (Intercept)            -0.2138*** (0.0184)                    
## high_earning           -0.1483*** (0.0295) -0.1011*** (0.0244)
## UG25abv                                                       
## high_earning x UG25abv                                        
## Fixed-Effects:         ------------------- -------------------
## institution_type                        No                 Yes
## ______________________ ___________________ ___________________
## S.E. type              Heteroskedast.-rob. Heteroskedast.-rob.
## Observations                           886                 886
## R2                                 0.02777             0.31493
## Within R2                               --             0.01831
## 
##                                         m4                  m5
## Dependent Var.:                ind_z_shift         ind_z_shift
##                                                               
## (Intercept)                                                   
## high_earning           -0.0985*** (0.0235)    -0.0418 (0.0352)
## UG25abv                -0.5085*** (0.0614) -0.3632*** (0.0996)
## high_earning x UG25abv                       -0.2033. (0.1051)
## Fixed-Effects:         ------------------- -------------------
## institution_type                       Yes                 Yes
## ______________________ ___________________ ___________________
## S.E. type              Heteroskedast.-rob. Heteroskedast.-rob.
## Observations                           882                 882
## R2                                 0.37027             0.37316
## Within R2                          0.09625             0.10040
## ---
## Signif. codes: 0 '***' 0.001 '**' 0.01 '*' 0.05 '.' 0.1 ' ' 1
\end{verbatim}

\begin{Shaded}
\begin{Highlighting}[]
\StringTok{"\_\_\_\_\_\_\_\_\_\_\_\_\_\_\_\_\_\_\_\_\_\_\_\_\_\_\_\_\_\_\_\_\_\_\_\_\_\_\_\_\_\_\_\_\_\_\_\_\_\_\_\_\_\_\_\_\_\_\_\_\_\_\_\_\_"}
\end{Highlighting}
\end{Shaded}

\begin{verbatim}
## [1] "_________________________________________________________________"
\end{verbatim}

\begin{Shaded}
\begin{Highlighting}[]
\FunctionTok{glht}\NormalTok{(m5, }\StringTok{"high\_earning + high\_earning:UG25abv = 0"}\NormalTok{) }\SpecialCharTok{\%\textgreater{}\%} \FunctionTok{summary}\NormalTok{()}
\end{Highlighting}
\end{Shaded}

\begin{verbatim}
## 
##   Simultaneous Tests for General Linear Hypotheses
## 
## Fit: feols(fml = ind_z_shift ~ high_earning + UG25abv + high_earning * 
##     UG25abv | institution_type, data = ., se = "hetero")
## 
## Linear Hypotheses:
##                                          Estimate Std. Error t value Pr(>|t|)
## high_earning + high_earning:UG25abv == 0 -0.24505    0.08204  -2.987   0.0029
##                                            
## high_earning + high_earning:UG25abv == 0 **
## ---
## Signif. codes:  0 '***' 0.001 '**' 0.01 '*' 0.05 '.' 0.1 ' ' 1
## (Adjusted p values reported -- single-step method)
\end{verbatim}

An interaction between the high-earning category and the percentage of
25+ aged students was included after reviewing the plot of student
percentage and post-scorecard implementation ranking shift, as it
appeared that a shift in the slope could potentially decrease the size
of the residuals of the linear regression line. Holding all over
variables constant, high-earning schools saw an increase in ranking
post-scorecard implementation of -0.04 standard deviations, with an
addition increase of -0.2 standard deviations when the percentage of 25+
aged students increased by 1 unit. As the result of the general linear
hypothesis test was statistically significant at a 1\% alpha, it was
concluded that the interaction should be dropped from future models.

\begin{Shaded}
\begin{Highlighting}[]
\NormalTok{m6 }\OtherTok{\textless{}{-}}\NormalTok{ scorecard\_f }\SpecialCharTok{\%\textgreater{}\%} \FunctionTok{feols}\NormalTok{(ind\_z\_shift }\SpecialCharTok{\textasciitilde{}}\NormalTok{ high\_earning }\SpecialCharTok{+}\NormalTok{ UG25abv }\SpecialCharTok{+}\NormalTok{ GRAD\_DEBT\_MDN\_SUPP}\SpecialCharTok{|}\NormalTok{ institution\_type, }\AttributeTok{se =} \StringTok{"hetero"}\NormalTok{)}
\end{Highlighting}
\end{Shaded}

\begin{verbatim}
## NOTE: 24 observations removed because of NA values (RHS: 24).
\end{verbatim}

\begin{Shaded}
\begin{Highlighting}[]
\FunctionTok{etable}\NormalTok{(m1, m2, m4, m6)}
\end{Highlighting}
\end{Shaded}

\begin{verbatim}
##                                     m1                  m2                  m4
## Dependent Var.:            ind_z_shift         ind_z_shift         ind_z_shift
##                                                                               
## (Intercept)        -0.2138*** (0.0184)                                        
## high_earning       -0.1483*** (0.0295) -0.1011*** (0.0244) -0.0985*** (0.0235)
## UG25abv                                                    -0.5085*** (0.0614)
## GRAD_DEBT_MDN_SUPP                                                            
## Fixed-Effects:     ------------------- ------------------- -------------------
## institution_type                    No                 Yes                 Yes
## __________________ ___________________ ___________________ ___________________
## S.E. type          Heteroskedast.-rob. Heteroskedast.-rob. Heteroskedast.-rob.
## Observations                       886                 886                 882
## R2                             0.02777             0.31493             0.37027
## Within R2                           --             0.01831             0.09625
## 
##                                       m6
## Dependent Var.:              ind_z_shift
##                                         
## (Intercept)                             
## high_earning         -0.0993*** (0.0237)
## UG25abv              -0.5129*** (0.0611)
## GRAD_DEBT_MDN_SUPP -7.98e-6*** (2.02e-6)
## Fixed-Effects:     ---------------------
## institution_type                     Yes
## __________________ _____________________
## S.E. type          Heteroskedastic.-rob.
## Observations                         862
## R2                               0.38526
## Within R2                        0.11772
## ---
## Signif. codes: 0 '***' 0.001 '**' 0.01 '*' 0.05 '.' 0.1 ' ' 1
\end{verbatim}

As a final step, the median debt of graduates was added to see if it had
an effect outside of the fixed effects of the institution type. The
result showed that holding all other variables constant, a one unit
increase in debt increased search rankings post scorecard by 0.000008
standard deviations. Given its minimal impact on the model, it was not
included in the final model.

\hypertarget{conclusion}{%
\subsection{Conclusion}\label{conclusion}}

\begin{Shaded}
\begin{Highlighting}[]
\FunctionTok{etable}\NormalTok{(m4)}
\end{Highlighting}
\end{Shaded}

\begin{verbatim}
##                                   m4
## Dependent Var.:          ind_z_shift
##                                     
## high_earning     -0.0985*** (0.0235)
## UG25abv          -0.5085*** (0.0614)
## Fixed-Effects:   -------------------
## institution_type                 Yes
## ________________ ___________________
## S.E. type        Heteroskedast.-rob.
## Observations                     882
## R2                           0.37027
## Within R2                    0.09625
## ---
## Signif. codes: 0 '***' 0.001 '**' 0.01 '*' 0.05 '.' 0.1 ' ' 1
\end{verbatim}

After reviewing the regression models, it was determined that the
strongest version controlled for percentage of students 25 years or
older, and institution type. In conclusion, high-earning colleges saw a
greater shift in interest following the release of the scorecard
compared to low-earning ones. The model shows that holding all other
variables constant, high-earning colleges saw an additional -0.1
increase in standardized search rankings post-publication over
low-earnings colleges.

\end{document}
